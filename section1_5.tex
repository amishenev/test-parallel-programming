\subsection{Автоматическое распараллеливание программ}

Параллельное программирование – достаточно сложный ручной процесс, поэтому кажется очевидной необходимость его автоматизировать с помощью компилятора. Такие попытки делаются, однако эффективность автораспараллеливания пока что оставляет желать лучшего, т.к. хорошие показатели параллельного ускорения достигаются лишь для ограниченного набора простых for-циклов, в которых отсутствуют зависимости по данным между итерациями и при этом количество итераций не может измениться после начала цикла. Но даже если два указанных условия в некотором for-цикле выполняются, но он имеет сложную неочевидную структуру, то его распараллеливание производиться не будет. Виды автоматического распараллеливания:

\begin{itemize}
    \item\textit{Полностью автоматический:} участие программиста не требуется, все действия выполняет компилятор.
    \item\textit{Полуавтоматический:} программист даёт указания компилятору в виде специальных ключей, которые позволяют регулировать некоторые аспекты распараллеливания.
\end{itemize}


Слабые стороны автоматического распараллеливания:

\begin{itemize}
    \itemВозможно ошибочное изменение логики программы.
    \itemВозможно понижение скорости вместо повышения.
    \itemОтсутствие гибкости ручного распараллеливания.
    \itemЭффективно распараллеливаются только циклы.
    \itemНевозможность распараллелить программы со сложным алгоритмом работы.
\end{itemize}

Приведём примеры того, как с-программа в файле src.c может быть автоматически распараллелена при использовании некоторых популярных компиляторов:

\begin{itemize}
    \itemКомпилятор GNU Compiler Collection:	 
    \mint[breaklines]{text}{gcc -O3 -floop-parallelize-all -ftree-parallelize-loops=K -fdump-tree-parloops-details src.c} 
    При этом программисту даётся возможность выбрать значение параметра K, который рекомендуется устанавливать равным количеству ядер (процессоров). Особенностям реализации автораспараллеливания в gcc посвящён самостоятельный проект: \url{https://gcc.gnu.org/wiki/AutoParInGCC}. 
    \itemКомпилятор фирмы Intel:  \mint[breaklines]{text}{icc -c -parallel -par-report file.cc}
    \itemКомпилятор фирмы Oracle: \mint[breaklines]{text}{solarisstudio -cc -O3 -xautopar -xloopinfo src.c}
\end{itemize}
