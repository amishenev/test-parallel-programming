{ %section4_2
	\subsection{Состав отчета}
	\begin{enumerate}
		\itemТитульный лист с названием вуза, ФИО студента и названием работы.
		\itemСодержание отчета (с указанием номера страниц и т.п.).
		\itemОписание решаемой задачи (взять из п.I и п.IV).
		\itemКраткая характеристика использованного для проведения экспериментов процессора, операционной системы и компилятора GCC (официальное название, номер версии/модели, разрядность, количество ядер, ёмкость ОЗУ, размер кэша и т.п.).
		\itemПолный текст программы lab1.c в виде отдельного файла.
		\itemТаблицы значений и графики функций seq(N), par-K(N) с указанием времени выполнения и величины параллельного ускорения. Предпочтительно использовать столбчатые гистограммы, показывающие зависимости времени или ускорения от размера массива.
		\itemПодробные выводы с анализом приведённых графиков и полученных результатов.
		\itemОтчёт предоставляется на бумажном носителе или на флешке.
	\end{enumerate}
}