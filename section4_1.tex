{ %section4_1
	\subsection{Порядок выполнения работы}
	\begin{enumerate}
		\itemНа компьютере с многоядерным процессором установить Unix-\linebreak подобную операционную систему и компилятор GCC версии не ниже 9.x. При невозможности установить Unix-подобную операционную систему или отсутствии компьютера с многоядерным процессором можно выполнять лабораторную работу на виртуальной машине. Минимальное число ядер при использовании виртуальной машины - два. Важным условием является отключение гипертрединга, для того, чтобы выполнить честные замеры времени. 
		\itemНа языке Cи написать консольную программу lab1.c, решающую задачу, указанную в п.4.4 (см. ниже). В программе нельзя использовать библиотечные функции сортировки, выполнения матричных операций и расчёта статистических величин. В программе нельзя использовать библиотечные функции, отсутствующие в стандартных заголовочных файлах stdio.h, stdlib.h, sys/time.h, math.h. Задача должна решаться 100 раз с разными начальными значениями генератора случайных чисел (ГСЧ). Структура программы примерно следующая:

        \inputminted[fontsize=\footnotesize]{c++}{listings/lab1Example.cpp}
        % \begin{figure}[H]
        %     \lstinputlisting{lab1Example.cpp}
        % \end{figure}

        \itemСкомпилировать написанную программу без использования автоматического распараллеливания с помощью следующей команды: /home/user/gcc -O3 -Wall -Werror -o lab1-seq lab1.c
		\itemСкомпилировать написанную программу, используя встроенное в gcc средство автоматического распараллеливания Graphite с помощью следующей команды  /home/user/gcc -O3 -Wall -Werror -floop-parallelize-all -ftree-parallelize-loops=K lab1.c -o lab1-par-K (переменной K поочерёдно присвоить хотя бы 4 значения: 1, меньше числа физических ядер, равное числу физических ядер и больше числа физических ядер).
		\itemВ результате получится одна нераспараллеленная программа и четыре или более распараллеленных.
		\itemЗакрыть все работающие в операционной системе прикладные программы (включая Winamp, uTorrent, браузеры, Telegram и Skype), чтобы они не влияли на результаты последующих экспериментов. При использовании ноутбука \textbf{необходимо иметь постоянное подключение к сети питания} на время проведения эксперимента.
		\itemЗапускать файл lab1-seq из командной строки, увеличивая значения N до значения N1, при котором время выполнения превысит 0.01 с. Подобным образом найти значение N=N2, при котором время выполнения превысит 5 с.
		\itemИспользуя найденные значения N1 и N2, выполнить следующие эксперименты (для автоматизации проведения экспериментов рекомендуется написать скрипт):
			\begin{itemize}
				\itemзапускать lab1-seq для значений \\$N\;=\;{N1,\;N1+\Delta,\;N1+2\Delta,\;N1+3\Delta,\dots,\;N2}$ и записывать получающиеся значения времени \texttt{delta\_ms(N)} в функцию \texttt{seq(N)};
				\itemзапускать lab1-par-K для значений \\$N\;=\;{N1,\;N1+\Delta,\;N1+2\Delta,\;N1+3\Delta,\dots,\;N2}$ и записывать получающиеся значения времени \texttt{delta\_ms(N)} в функцию \texttt{par-K(N)};
				\itemзначение $\Delta$ выбрать так: $\Delta\;=\;(N2\;-\;N1)/10$.
			\end{itemize}
		\itemПровести верификацию значения X. Добавить в конец цикла вывод значения X и изменить количество экспериментов на 5. Сравнить значения X для распараллеленной программы и не распараллеленной.
		\itemНаписать отчёт о проделанной работе.
		\itemПодготовиться к устным вопросам на защите.
		\itemНайти вычислительную сложность алгоритма до и после распараллеливания, сравнить полученные результаты.
		\sloppy
		\item\textbf{Необязательное задание №1 (для получения оценки «четыре» и «пять»).} Провести аналогичные описанным эксперименты, используя вместо gcc компилятор Solaris Studio (или любой другой на своё усмотрение). При компиляции следует использовать следующие опции для автоматического распараллеливания: \mint{console}{solarisstudio -cc -O3 -xautopar -xloopinfo lab1.c}
     \item\textbf{Необязательное задание №2 (для получения оценки «пять»).} Это задание выполняется только после выполнения предыдущего пункта. Провести аналогичные описанным эксперименты, используя вместо gcc компилятор Intel ICC (или любой другой на своё усмотрение). В ICC следует при компиляции использовать следующие опции для автоматического распараллеливания: \mint[breaklines]{console}{icc -parallel -par-threshold=0 -par-num-threads=K -o lab1-icc-par-K lab1.c}
	\end{enumerate}
	
}
